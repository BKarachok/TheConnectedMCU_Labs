In this project, a lighting control system with IoT is implemented. The device consists of a chipkit Wi\+Fire board with P\+I\+C32\+MZ microcontroller, bluetooth module H\+C-\/05, light sensor B\+H1750. The lamp uses the L\+D1 L\+ED on the board. Control is carried out using the Android application on the smartphone via bluetooth. The screen displays the value of the illumination level, which is measured by the sensor. There are 2 modes of control -\/ manual and smart. In manual mode, you can turn the lamp on and off with the buttons. To activate smart mode, you must press the Smart Mode button. In smart mode, the lamp is on if the illumination level is less than 20 lux, otherwise the lamp is turned off.

For programming, the Arduino I\+DE was used. To configure the Arduino I\+DE, you need to follow the instructions in the guides\+: \href{https://www.youtube.com/watch?v=DOEdmc57FVU}{\tt https\+://www.\+youtube.\+com/watch?v=\+D\+O\+Edmc57\+F\+VU} and \href{https://chipkit.net/wiki/index.php?title=ChipKIT_core}{\tt https\+://chipkit.\+net/wiki/index.\+php?title=\+Chip\+K\+I\+T\+\_\+core}. You also need to download and connect the libraries\+: \hyperlink{math_8h}{math.\+h}, \hyperlink{_wire_8h}{Wire.\+h}, \hyperlink{_w_string_8h}{W\+String.\+h}. 